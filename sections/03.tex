\section{数据表示:表格}
\label{sec:tables}
表格是整理和展示结构化数据的常用工具。我们使用 \texttt{booktabs} 宏包来创建更专业的横线(\texttt{\string\toprule}, \texttt{\string\midrule}, \texttt{\string\bottomrule})。

\begin{table}[h]
    \centering
    \caption{随机生成的数据集摘要}
    \label{tab:summary}
    \begin{tabular}{lccc}
        \toprule
        \textbf{参数} & \textbf{值 A} & \textbf{值 B} & \textbf{差异} \\
        \midrule
        样本量 $(N)$ & 120 & 150 & -30 \\
        平均值 $(\mu)$ & 45.7 & 48.1 & 2.4 \\
        标准差 $(\sigma)$ & 5.2 & 4.9 & -0.3 \\
        \bottomrule
    \end{tabular}
    \vspace{0.5em} % 增加一些垂直空间
    \footnotesize 注:差异计算为 值 B - 值 A。
\end{table}
从表 \ref{tab:summary} 可以看出,尽管样本量不同,值 B 的平均值略高于值 A。

\section{进阶数学环境}
\label{sec:advanced-math}
本节展示矩阵和分段函数。

\subsection{矩阵示例}
一个 $3 \times 3$ 的系数矩阵 $A$ 可以使用 \texttt{pmatrix} 环境来表示(带小括号):
$$A = \begin{pmatrix}
    a_{11} & a_{12} & a_{13} \\
    a_{21} & a_{22} & a_{23} \\
    a_{31} & a_{32} & a_{33}
\end{pmatrix}$$
我们也可以使用 \texttt{bmatrix} 环境(带中括号)来表示一个向量 $v$:
$$v = \begin{bmatrix} 1 \\ 0 \\ -1 \end{bmatrix}$$

\subsection{分段函数}
分段函数 $f(x)$ 使用 \texttt{cases} 环境来定义:
$$f(x) =
\begin{cases}
    x^2 + 1, & \text{如果 } x < 0 \\
    5, & \text{如果 } x = 0 \\
    \frac{1}{x}, & \text{如果 } x > 0
\end{cases}$$

